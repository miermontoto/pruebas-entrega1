\chapter{Reporte de fallos}
% ordenados por severidad
Tras la ejecución del diseño de pruebas anterior, se obtiene la siguiente lista
de fallos, ordenados por severidad. La \textit{severidad} que se le asigna a
cada fallo es totalmente subjetiva y se basa en la importancia que tiene el
mismo en el correcto funcionamiento de la aplicación.

Primero de todo, se ha de tener en cuenta los fallos de truncamiento que se producen
al introducir números decimales en los campos de entrada, algo fundamental que
provoca salidas erróneas.

También se considera grave el fallo que causa el incumplimiento de la desigualdad
triangular cuando la suma de los dos últimos lados es inferior que el primero.
Este fallo lógico se considera el grave ya que incumple claramente una de las
propiedades fundamentales de los triángulos.

El fallo que provoca el cierre inesperado de la aplicación al introducir un
caracter alfabético en uno de los campos debería corregirse con cierta urgencia,
puesto que es un detalle esencial que se debería haber comprobado durante el
desarrollo y que cualquier usuario puede cometer fácilmente.

Otro de los detalles que se deberían haber tenido en cuenta es el resultado de
introducir tres ceros en los campos de entrada, ya que se produce ``equilátero''
cuando, evidentemente, se trata de un triángulo inválido.

El hecho de que la aplicación considere como ``rectángulos'' a los triágulos que
los son no es un fallo lógico ni técnico, pero era un detalle que se debería haber
estipulado durante la fase de diseño.
