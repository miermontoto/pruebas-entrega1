\chapter{Diseño y resultados de la ejecución}
% identificador, objetivo, entradas, resultados esperados, resultados obtenidos (colores ok)
En este capítulo se describe el diseño de la aplicación de juguete
y se muestran los resultados de la ejecución de la misma.

\section{Situaciones de prueba}
Para probar el correcto funcionamiento de la aplicación,
se crean las siguientes situaciones de prueba que tratan de cubrir
todos los casos de uso posibles de la aplicación.

\begin{itemize}
	\item \textbf{Datos}
		\begin{enumerate}[label=D\arabic*.]
			\item Números
				\begin{enumerate}[label=D1.\arabic*.]
					\item Naturales
					\item Negativos
					\item Decimales
						\begin{enumerate}[label=D1.3.\arabic*.]
							\item Separación válida (punto)
							\item Separación inválida (coma)
							\item Imprecisión por aproximación
							\item Imprecisión por truncamiento
						\end{enumerate}
					\item Cero
						\begin{enumerate}[label=D2.3.\arabic*.]
							\item Cero en el $1^{er}$ campo
							\item Cero en el $2^{do}$ campo
							\item Cero en el $3^{er}$ campo
							\item Cero en todos los campos
						\end{enumerate}
				\end{enumerate}
			\item Otros
				\begin{enumerate}[label=D2.\arabic*.]
					\item Letras
					\item Caracteres raros (no alfanuméricos)
					\item Campos vacíos
				\end{enumerate}
		\end{enumerate}
	\item \textbf{Lógica}
		\begin{enumerate}[label=L\arabic*.]
			\item Triángulos válidos
				\begin{enumerate}[label=L1.\arabic*.]
					\item Equilátero
					\item Isósceles
					\item Escaleno
					\item Rectángulo (no contemplado)
				\end{enumerate}
			\item Incumplimiento de la desigualdad triangular
				\begin{enumerate}[label=L2.\arabic*.]
					\item Menor que la suma
						\begin{enumerate}[label=L2.1.\arabic*.]
							\item $a + b < c$
							\item $b + c < a$
							\item $c + a < b$
						\end{enumerate}
					\item Igual que la suma
						\begin{enumerate}[label=L2.2.\arabic*.]
							\item $a + b = c$
							\item $b + c = a$
							\item $c + a = b$
						\end{enumerate}
				\end{enumerate}
		\end{enumerate}
	\item \textbf{Técnico}
		\begin{enumerate}[label=T\arabic*.]
			\item Overflow
				\begin{enumerate}[label=T1.\arabic*.]
					\item Por \Verb#short# (16 bits, $>32.767$)
					\item Por \Verb#int# (32 bits, $>2.147.483.647$)
					\item Por \Verb#long# (64 bits, $>9.223.372.036.854.775.807$)
				\end{enumerate}
			\item Underflow
				\begin{enumerate}[label=T2.\arabic*.]
					\item Por \Verb#short# (16 bits, $<(-32.767)$)
					\item Por \Verb#int# (32 bits, $<(-2.147.483.647)$)
					\item Por \Verb#long# (64 bits, $<(-9.223.372.036.854.775.807)$)
				\end{enumerate}
		\end{enumerate}
\end{itemize}
\newpage{}
\section{Casos de prueba y ejecución}
Para comprobar el correcto funcionamiento de la aplicación en todas las
situaciones anteriores, se definen los siguientes casos de prueba que
deberán probar cada una de las situaciones anteriores, teniendo en cuenta
siempre que un caso de prueba puede cubrir más de una situación siempre y
cuando que no se traten de situaciones inválidas.

Antes de la ejecución se anota el resultado esperado, con el que posteriormente
se comparará el resultado obtenido.

\begin{table}[ht]
	\centering
	\rowcolors{2}{gray!15}{white}
	\begin{tabular}{|c|c|c|c|}
		\hline
		\textbf{Entrada} & \textbf{Situaciones cubiertas} & \textbf{Salida esperada} & \textbf{Resultado} \\
		\hline
		2, 2, 3 & D1.1, L1.2 & Isósceles & \cellcolor{green!25} OK \\
		2.2, 2.2, 2.2 & D1.3.1, L1.1 & Equilátero & \cellcolor{green!25} OK \\
		2,2, 2,2, 2,2 & D1.3.2 & Inválido & \cellcolor{green!25} OK \\
		2.99, 3, 3 & D1.3.3 & Isósceles & \cellcolor{green!25} OK \\
		2.01, 2, 2 & D1.3.4 & Isósceles & \cellcolor{red!25} MAL (equilátero) \\
		-1, 3, 4 & D1.2 & Inválido & \cellcolor{green!25} OK \\
		a, 2, 2 & D2.1 & Inválido & \cellcolor{red!25} MAL (se cierra) \\
		ø, 3, 4 & D2.2 & Inválido & \cellcolor{green!25} OK \\
		~, 2, 3 & D2.3 & Inválido & \cellcolor{green!25} OK \\
		0, 2, 3 & D2.3.1 & Inválido & \cellcolor{green!25} OK \\
		1, 0, 3 & D2.3.2 & Inválido & \cellcolor{green!25} OK \\
		1, 2, 0 & D2.3.3 & Inválido & \cellcolor{green!25} OK \\
		0, 0, 0 & D2.3.4 & Inválido & \cellcolor{red!25} MAL (equilátero) \\
		2, 3, 4 & L1.3 & Escaleno & \cellcolor{green!25} OK \\
		3, 4, 5 & L1.4 & Escaleno & \cellcolor{red!25} MAL (rectángulo) \\
		1, 1, 5 & L2.1.1 & Inválido & \cellcolor{green!25} OK \\
		5, 1, 2 & L2.1.2 & Inválido & \cellcolor{red!25} MAL (escaleno) \\
		1, 6, 2 & L2.1.3 & Inválido & \cellcolor{green!25} OK \\
		2, 2, 4 & L2.2.1 & Inválido & \cellcolor{green!25} OK \\
		2, 1, 1 & L2.2.2 & Inválido & \cellcolor{green!25} OK \\
		1, 3, 2 & L2.2.3 & Inválido & \cellcolor{green!25} OK \\
		-33000 (x3) & T1.1 & Inválido & \cellcolor{green!25} OK \\
		-2150000000 (x3) & T1.2 & Inválido & \cellcolor{green!25} OK \\
		-10000000000000000000 (x3) & T1.3 & Inválido &\cellcolor{green!25} OK  \\
		33000 (x3) & T2.1 & Equilátero & \cellcolor{green!25} OK \\
		2150000000 (x3) & T2.2 & Equilátero & \cellcolor{green!25} OK \\
		10000000000000000000 (x3) & T2.3 & Equilátero & \cellcolor{green!25} OK \\
		\hline
	\end{tabular}
\end{table}
